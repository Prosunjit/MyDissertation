\begin{abstract}
	
Logical formulas and enumeration  are two major ways for specifying authorization policies in an Attribute Based Access Control (ABAC) model.  In this paper, we show that for finite domain attributes, logical-formula and enumerated authorization policy ABAC models are equivalent in theoretical expressive power. We also analyze these models beyond their theoretical expressive power. While logical-formulas are known to be flexible and convenient for setting up new authorization  policies, we demonstrate that they are difficult to update/administer policies. On the other hand, enumeration  can be difficult to set up new policies but they are convenient for  updating and administering existing policies. We show enumerated policies support micro-policies and can easily be represented in a canonical form. Canonical representation and support for micro-policies enables automated and scalable administration. 
	
%Enumerated and logical-formula  policy are two major ways for specifying authorization policies in Attribute Based Access Control (ABAC). While, logical-formula authorization policy ABAC models (\LPModels) have been studied relatively well, enumerated authorization policy models (\EPModels) are still immature. In this paper, we present a finite attribute, finite domain model for \EPModels{} and understand its relationship with \LPModels{}. We find that in term of expressive power  \EPModels{} and \LPModels{} are equivalent regardless of using single or multiple attributes to represent users and objects. We show that in \EPModels{}, an authorization policy is decomposed of micro-policies and can easily be represented in a canonical form. The implications are policy review is easy   in \EPModels{} (as opposed NP-complete in \LPModels{}), policy update can be automated (as opposed to manual in \LPModels{}) and policy administration would become more scalable.
\end{abstract}