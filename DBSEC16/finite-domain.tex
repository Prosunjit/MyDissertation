Most of the ABAC models (for example, \cite{abacAlpha,hgabac,abac-ws,abac-for-web-service}) assume finite set of user and  object attributes and that values of each attribute come from a finite set. This assumption is useful in many practical cases. For example, values of \textit{roles}, \textit{clearance} or \textit{age} are bounded and mostly static. But attribute values can be unbounded as well. For example, if values of an attribute include users or objects in a system  (e.g. \textit{owner} of an object) where they may grow indefinitely, these values are unbounded.  In this paper, we assume that there is a finite set of attributes and values of each attribute come from a finite set. Without this finiteness assumption, it is not possible to enumerate logical formulas.

%Nonetheless, the assumption of \textit{finite domain } is at the very core of the equivalence results presented in this paper. Informally, we define a finite domain ABAC model as a model having bounded set of attributes and each attribute having bounded set of values.

%For the sake of clarity, we scope our work for finite domain ABAC  models where exists set of attributes and each attribute has s finite set of values.


%For example, if we consider  \textit{owner} as an object attribute and its value can be any user in the system, range of this attribute may not be a  finite set.

 %Nonetheless, the assumption of \textit{finite domain ABAC} is at the very core of the results presented in this paper. For the sake of clarity, we scope our work for finite domain ABAC  models where exists set of attributes and each attribute has s finite set of values.
%\vspace{-1em}