Logical-formula authorization policy (\LAP{}) can be defined as a boolean expression consisting of subexpressions connected with logical operators (for example, $\land, \lor, \lnot$ and so on ) where each subexpression compares attribute values with other attribute or constant values. The language for \LAP{} usually supports a large set of logical and relational operators. A \LAP{} grants a user request for exercising certain action on an object if attributes of the requesting user and requested object evaluate the formula true. $Auth_{read} \equiv clearance(u) \succeq classification(o)$ is an example of  logical-formula authorization policy which allows a user to read an object if the user's clearance dominates classification of the object.

\LAP{}s are usually expressed in propositional logic. Examples of  \LPModels{} models  include \cite{abacAlpha,hgabac,abac-ws,abac-for-web-service}.  Flexibility of these models have been demonstrated by configuring conventional DAC\cite{dac}, MAC\cite{lbac} and RBAC \cite{rbac} policies in it. It has been shown  \cite{labac} that policy review in  \LPModels{} is equivalent to the satisfiability problem  which is NP-complete for propositional logic. 
\vspace{-1em}
%in proposition logic (or in first-order logic if LAPs are expressed in it) which makes the complexity of reviewing policies NP-complete in \LPModels{} models.