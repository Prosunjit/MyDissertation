\section{Introduction}
Attribute Based Access Control (ABAC) has gained considerable attention from businesses, academia and standard bodies, such as NIST \cite{nist-abac-draft}, in recent years. ABAC uses attributes on users, objects and possibly other entities (e.g. context/environment) and specifies rules using these attributes to assert who can have which access permissions (e.g. read/write) on which objects.  Although ABAC concepts have been around for over two decades there remains a lack of well-accepted ABAC models.  Recently there has been a resurgence of interest in ABAC due to continued dissatisfaction with the three traditional models (DAC \cite{dac}, MAC \cite{lbac}, RBAC \cite{rbac}), particularly the limitations of RBAC.

To demonstrate expressive power and flexibility, several ABAC models including  \cite{abacAlpha,hgabac,abac-ws,abac-for-web-service} have been proposed in past few years. These models adopt the conventional approach of designing attribute based authorization policies as logical formulas. Logical formulas are convenient to specify authorization policies  because they are easy to set up and powerful  to specify even complicated business requirements in a concise way. 
 

Another method for specifying authorization policies in ABAC is by enumeration. Examples in this category include Policy Machine (PM) \cite{policy-machine} and  \labac{} \cite{labac}. Both \PM{} and \labac{} demonstrate enumerated policies are also very expressive by their ability to configure traditional models. 

Thus, enumeration and logical-formulas are two viable approaches to express authorization policies in an ABAC model. While logical-formula authorization policy ABAC (denoted \LPModels{}) models have received considerable attention for some time, design and development of enumerated authorization policy ABAC models (denoted \EPModels) is relatively recent. As a result, there has been very few work that compare and contrast these two approaches. 

%Authors in \cite{labac} specifies that policy review problem in LAP is equivalent to  satisfiability problem in the language policy is specified. Thus, policy review in most LAPs are     NP-Complete because they are often represented in propositional logic (e.g. \cite{abacAlpha,hgabac}).  The authors do not specify complexity of policy  review in EAPs. 

%In this paper, we analyze logical-formula and enumerated authorization policy ABAC models (\LPModels{} and \EPModels{}) in finite domain.

In this paper, we present a finite attribute, finite domain model for \EPModels{} and investigate its relationship with \LPModels{} in the finite domain. We first analyze theoretical expressive power of these two category of models. We show that \LPModels{} and \EPModels{} models are theoretically equally expressive. In this process, we also show that single-attribute and multi-attribute ABAC models are also theoretically equivalent. 

We have further investigated these models beyond their theoretical expressive power.  We find that \LPModels{}s are flexible for setting up new authorization polices whereas \EPModels{}s are flexible for updating or administering policies. One observation is that while sub policies in a logical formula authorization policy (denoted  LAP) are tightly coupled, an enumerated authorization policy (denoted EAP) is composed of micro-policies which can be considered as the minimum administrative unit in \EPModels{}.  We additionally show that EAPs can be easily  represented in a canonical form which is convenient to represent, update or administer policies. 

Rest of this paper is organized as follows. In Section \ref{sec:background}, we briefly discuss different styles and scopes of ABAC models. In Section \ref{sec:models}  we present \EPModels{} and \LPModels{} models respectively. We show that these models are equivalent in theoretical expressive power in Section \ref{sec:equivalence}. In Section \ref{sec:beyond}, we discuss these models beyond their expressive power. We present related work in Section \ref{sec:related-work}. Finally, we conclude the paper in Section \ref{sec:conclusion}. 