\newcommand{\phoneRE}{\textit{RE-for-phone}}
\newcommand{\emailRE}{\textit{RE-for-email}}
\newcommand{\phoneNumbers}{\textit{phone-numbers}}
\begin{table}[t]
	\centering
	\caption{ Example of content based labeling policies} %\vspace*{3pt}
	\label{tab:labac-definition}			
	\begin{tabular}{|l|}
		\hline					
		\begin{tabular}{l}
				%\multicolumn{1}{c}{\underline{\textit{I. Assignment of security-label values on JSON objects}}}\\									
				
					- $RE$= \{\phoneRE, \emailRE \} . \\
					- $\matchingElements$ = $\{(Value,\phoneRE)\},  \{ (Key, ``SSN"),$ \\ \hfil  $(Value, \emailRE)\}$. \\
					- $TO = \{ \phoneNumbers \equiv \{(Value,\phoneRE)\}, emails \equiv  \{(Value, \emailRE)\} \}$\\
					- $SCOPE = \{sc1\equiv (\match, \oneLevelUP, \unrestrictedLabeling),$ \\ \hfil $  sc2\equiv (\exactMatch, \noProp, \unrestrictedLabeling)\}$\\
					- $SL = \{ sl1, sl2, sl3, sl4\} $ \\				
					- $\labelAssignment = \{ (\phoneNumbers, sc1, \{sl1\}, (emails, sc2, \{sl2\} ) ) \}$\\					
							
		\end{tabular}			
		 \\ \hline	
	\end{tabular}
	
\end{table}

