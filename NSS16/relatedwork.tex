\section{Background and related work}
\label{sec:related-work}

In this section, we briefly review JSON and discuss related work.

\chapter{Background \& Literature Review}

\section{Finite Domain ABAC}
	Most of the ABAC models (for example, \cite{abacAlpha,hgabac,abac-ws,abac-for-web-service}) assume finite set of user and  object attributes and that values of each attribute come from a finite set. This assumption is useful in many practical cases. For example, values of \textit{roles}, \textit{clearance} or \textit{age} are bounded and mostly static. But attribute values can be unbounded as well. For example, if values of an attribute include users or objects in a system  (e.g. \textit{owner} of an object) where they may grow indefinitely, these values are unbounded.  In this dissertation, we assume that there is a finite set of attributes and values of each attribute come from a finite set.
	
\section{Types of authorization policy}
	Most of the ABAC models assume a finite set of user attributes, finite set of object attributes and finite range for each of these attribute functions. On the other hand, when specifying authorization policies, there are two major methods. More conventional approach is to define policies using logical formula. Examples in this category include $\abacAlpha${} \cite{abacAlpha}, \hgabac{} \cite{hgabac}, ABAC for Web Services \cite{abac-for-web-service}, and XACML \cite{xacml}. The alternative technique for expressing policy is by enumeration. Examples in this category include Policy Machine (PM) \cite{policy-machine} and \twoSortedRBAC{} \cite{two-sorted-rbac}.
	
\subsection{Logical-formula Authorization policy}
	Logical-formula authorization policy (\LAP{}) can be defined as a boolean expression consisting of subexpressions connected with logical operators (for example, $\land, \lor, \lnot$ and so on ) where each subexpression compares attribute values with other attribute or constant values. The language for \LAP{} usually supports a large set of logical and relational operators. A \LAP{} grants a user request for exercising certain action on an object if attributes of the requesting user and requested object evaluate the formula true. $Auth_{read} \equiv clearance(u) \succeq classification(o)$ is an example of  logical-formula authorization policy which allows a user to read an object if the user's clearance dominates classification of the object.
	
	\LAP{}s are usually expressed in propositional logic. Examples of  \LPModels{} models  include \cite{abacAlpha,hgabac,abac-ws,abac-for-web-service}.  Flexibility of these models have been demonstrated by configuring conventional DAC\cite{dac}, MAC\cite{lbac} and RBAC \cite{rbac} policies in it. 
	
	%It has been shown  \cite{labac} that policy review in  \LPModels{} is equivalent to the satisfiability problem  which is NP-complete for propositional logic. 
	
	As satisfiability in propositional logic is NP-complete and policy review in general can be mapped to satisfiability problem, reviewing policy would be NP-Complete in many existing ABAC models including \cite{abacAlpha,hgabac,abac-for-web-service}. On the other hand, if policies are expressed in first-order logic, policy review would be undecidable since  satisfiability is undecidable in first-order logic.
	
\subsection{Enumerated Authorization Policy}
	Usefulness of enumerated authorization policy has been demonstrated in the literature. For example, in Policy Machine (PM) \cite{policy-machine}, Ferraiolo et. al define attribute based enumerated policies using one user attribute, one object attribute and a set of actions. A policy/privilege in PM is defined as $(ua_i, OP, oa_i)$, where $ua_i$ and $oa_i$ are values of user-attribute and object-attribute respectively and $OP$ is  a set of operations. Intuitively, reviewing or updating an enumerated policy would be polynomial time.
		
	The simple structure of enumerated policy does not necessarily make it less expressive. For example, PM shows how to configure traditional models using enumerated policies \cite{INCITS526}. In Section \ref{sec:configuration}, we also show how to express RBAC \cite{rbac} and LBAC \cite{lbac} policies using \EAP{}s. Furthermore, in Section \ref{sec:equivalence}, EAP and LAP are are equivalent in their theoretical expressive power.
	
	Informally, an enumerated authorization policy (\EAP{}) consists of a set of  tuples.  Each tuple \textit{(UAVals, OAVals)} grants privileges to a set of users  to exercise an action on a set of objects identified by the user and object attribute values \textit{UAVals} and \textit{OAVals} respectively. In an EAP, each tuple is distinct and grants privileges independently. Both  UAVals{} and OAVals{} can be atomic valued or set valued. (\textit{\manager, TS}) and (\textit{\{\manager, dir\}, \{TS,H\}}) are example of atomic and set valued tuples respectively. 
	
	

%\section{Theoretical expressive power}

\section{Literature Review}


Several attribute based access control models have been proposed in the literature. While, some authors design general purpose ABAC model, others design ABAC in specific application context. There are also significant works towards integrating attributes with traditional RBAC model for enhancing its expressibility. Furthermore, XACML represent another line of work involving attributes to provide flexible policy language and support of multiple access control policies.

$\abacAlpha${} \cite{abacAlpha} is among the first few models to formally define an ABAC model. It is designed to demonstrate flexibilities of an ABAC system to configure DAC, MAC and RBAC models. $\abacAlpha${} uses subset of subject attributes and object attributes to define  an authorization policy for a particular permission $p$. It describes a constraint language to specify subject attributes from user attributes. Furthermore, it also presents a constraint language for changing object attributes at  creation or modification time.

\hgabac{} \cite{hgabac} is another notable work in designing a formal model for an ABAC system. Besides designing a flexible policy language capable of  configuring DAC, MAC and RBAC, it also addresses a real problem of assigning attributes to a large set of users and objects. It specifies hierarchical groups and provides a mechanism for inheriting attributes from a group by joining to the group.

ABAC-for-web-services \cite{abac-for-web-service} is among very few earlier works to outline authorization architecture and policy formulation for an ABAC system. They propose a distributed architecture for authoring, administering, implementing and enforcing an ABAC system. Even though, their policy language is semi-formal, they present a powerful idea of composing hierarchical policies from individual policies.

Wang et al \cite{wang2004logic} presents a stratified logic programming based framework to specify ABAC policies. Even though, they only consider user attributes, they focus on providing a consistent, high performance and workable solution for ABAC system.



In its ABAC guide \cite{nist-abac-draft} and other publications \cite{hu2015attribute},  NIST defines common terminologies, and concepts for an ABAC system. It discusses required components, considerations and architecture for designing an  enterprise ABAC system. It acknowledges the fact that ABAC rules can be quite complex in boolean combination of attributes or in simple relations involving attributes. Additionally, it discusses more advanced features like attribute and policy engineering, federation of attributes and so on. Nonetheless, these documents are focused towards establishing general definitions and considerations of an ABAC system without providing a concrete model definition.

There are other works that design an ABAC system from a particular application context. For example, WS-ABAC \cite{abac-ws}  is motivated by requirements in web services, ABAC-in-grid \cite{grid-abac}  is motivated by needs in the grid computing.



Another interesting line of work combines attributes with Role Based Access Control. Kuhn et. al \cite{kuhn2010adding} provides a framework for combining roles and attributes. In the framework, they briefly outline three different approaches - (i) dynamic roles which retain basic structure or RBAC and  uses attribute based rules to derive user roles, (ii) attribute centric,  which treat role as another ordinary attribute, and (iii) role centric, which uses roles to grant permissions and attributes to reduce permissions to be available to the user. Various other earlier or subsequent works involving roles and attributes can also be cast in Kuhn's framework. For example, attribute-based user-role assignment by Al-Kahtani et. al \cite{al2002model} can be considered as an approach based on dynamic roles.

Last but not the least, XACML \cite{xacml} is a declarative access control policy language and processing model which  supports attribute based access control concepts and policies. Although, it lacks a formal definition of an ABAC model, it is notable for its uses in multiple commercial products.
%\clearpage


\subsection{Related work}

There is limited academic research published  on security of  JSON data. To the best of our knowledge, we are the first to propose a protection model for it.

On the other hand, XML security has long been investigated by many researchers. A fundamental line of work in this area is about specifying authorization policies for the protection of XML documents \cite{policy-based4,policy-based2,policy-based5,policy-based6}. All of these models attach authorization policies directly on nodes in the XML tree. Most of these models use XPath \cite{Xpath} to specify a node in the tree. For example, Damiani et al. \cite{damiani2002fine} specify authorization policies as a tuple of $\langle$\textit{subject, object, action, sign, type} $\rangle$ where  an \textit{object} is identified by an URI (Uniform Resource Identifier) along with a XPath  expression.

 %On the other hand, XML security has long been investigated by many researchers.  The fundamental works on XML authorization are about developing fine grained access control models for XML. Notable works in this area include \cite{policy-based1,policy-based2,policy-based4,policy-based5,policy-based6}. All of these works  specify XML nodes (both XML tag and attributes) in the XML tree as the minimum unit of protection and specify access control policies for the nodes using XPath or similar path based expression.  For example, Damiane et. al specify authorization policies as a tuple of (subject, object, action, sign, type) where  object is identified by an URI (Uniform Resource Identifier) and an XPath \cite{Xpath} expression.

Another direction of work is about effective enforcement of authorization mechanisms  for secure and efficient query evaluation. For example, in \cite{xml-view1} the authors derive \textit{security views} comprising exactly the set of accessible nodes for different user groups. Based on the security view, they  provide a unique DTD view for each user group. Similar works in this direction include  \cite{xml-performance2,xml-performance1} which use query preprocessing approaches. These models uses preprocessed finite automatas for authorization policies, document and Schema/DTD, and determine if a query is safe before running it. Unsafe queries can be rewritten.

%Other notable works in this area includes \cite{??}.

The idea of associating labels with protected objects has been proposed before. For example,  in \textit{purpose based access control (PBAC)}~\cite{purpose-bac}, the authors associate \textit{intended purposes} with data items and \textit{access purpose} with users. If \textit{access purpose} of a user is included in the \textit{intended purposes} of the requested objects, the request is granted. Our approach is similar. While PBAC manages intended purposes using RBAC \cite{rbac}, we use attributes with attribute-based access control (ABAC). Most significantly, PBAC does not specify how to annotate objects with \textit{access purposes}, which we emphasize in this paper via labeling policies. Adam et al. \cite{content-bac}, have applied \textit{concepts} and \emph{slots} on digital objects which work at a finer grained content level. They have also   specified an access control model based on expressions using concepts and slots.  This model also does not specify how to assign concepts and slots to objects. 

The concept of attaching organized labels to users and objects and controlling access based on these labels is the underlying idea of Lattice Based Access Control (LBAC \cite{lbac}), sometime also referred as Mandatory Access Control (MAC). The operational model, \atom{}, presented in Section \ref{sec:operational-model} resembles LBAC but it is fundamentally different from LBAC. \atom{} is based on enumerated authorization policy ABAC model named \eapabac{} (\cite{labac,eap-abac}). \eapabac{} is a general purpose ABAC model which supports larger set of attributes contrary to single label in LBAC and based on enumerated authorization policies. Correlation between \eapabac{} and LBAC is presented in \cite{labac}.


