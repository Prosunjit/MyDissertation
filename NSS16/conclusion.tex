\section{Conclusion}
\label {sec:conclusion}

This paper presents an attribute based protection model for JSON documents. In the proposed model, JSON elements are annotated with \textit{security-label} attribute values with \textit{labeling policies}. We specify \textit{authorization policies} using these attribute values. The advantage of the separation of labeling and authorization policies is that they can be specified and administered independently possibly by different level of administrators. In this regard, we have presented an operational model to specify authorization policies that evaluates access  request. Further, we have  specified two different models for assigning security-label attribute values on JSON elements based on content and paths. We have presented a proof-of-concept of the proposed models in OpenStack IaaS cloud platform.

%One advantage of this approach is that authorization policies can be administered independently by the higher level administrators  without knowing details of the JSON structure. On the other hand, labeling policies to assign attribute values to JSON documents can be managed by local administrators.  We have presented the formal definition for the operational model for evaluating access requests and two different administrative models (based on JSON content and JSON path) for assigning \textit{security-label} attribute values. Finally, we have demonstrated the feasibility of our proposed models by implementation of them in OpenStack platform.