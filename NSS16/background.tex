\subsection{JSON (JavaScript Object Notation)}
\label{sec:background}


JSON or JavaScript Object Notation is a format for representing textual data in a structured way.  In JSON, data is represented in one of two forms --- as an \emph{object} or an \emph{array} of values. A JSON object is defined as a collection of \emph {key,value} pairs where a \emph {key} is simply a string representing a name  and a \emph {value}  is one of the following primitive types---string, number, boolean, null or another object or an array. The definition of a JSON object is recursive in that an object may contain other objects. An array is defined as a set of an ordered collection of  values. JSON data  manifests following characteristics.

%\vspace{-1.0em}
\begin{itemize}

  \item JSON data forms a rooted tree hierarchical structure. 

  \item In the tree, leaf nodes represent values and a non-leaf nodes represent keys.

  \item A node in the tree, can be uniquely identified by a unique path.
\end{itemize}

Figure \ref{fig:JSON-example}(a) shows the content of a JSON document where strings representing values have been replaced by \textit{``..."} for ease of presentation.  Figure \ref{fig:JSON-example}(b) shows the corresponding tree representation. Any node in the tree can be uniquely represented by  JSONPath \cite{JSONPath} which is a standard representation of paths for JSON documents.





	\begin{figure} [t]
 		\centering
 		\includegraphics[width=1\textwidth]{JSON-example}
 		\caption{Example of (a) JSON data (b) corresponding JSON tree}
 		\label{fig:JSON-example}
 	\end{figure}

%\subsubsection{JSONPath}








%\vspace{-.5em}



%
%\subsubsection{JSON vs XML}
%\label{sec:json-vs-xml}
%
%Syntactically XML has a different structure than JSON. But both of them represent a common hierarchical structure. Thus authorization policy specified for JSON data can also be used for XML and vice versa. 



%\subsection{LaBAC as an enumerated ABAC model}



