\subsection{Motivation}
OpenStack  Swift is a highly deployed open source cloud storage solution. With its unlimited storage capacity, it is used to store any number of large or small objects. In Swift terminology, uploaded documents are called objects.  A user can  upload or download an object using well defined APIs or available Swift client programs. But not everyone can download (i.e. read) every object stored in Swift. In order to maintain who can or cannot access an object, Swift uses Access Control Lists (ACLs). ACL specifies who can or cannot access an object. Unfortunately, the ACL based approach for Swift is an `all or nothing' approach in a way that an user can either download (i.e. read) the whole object or cannot download it at all.

We propose a content level access control mechanism for objects  stored in Swift.  This approach lets Swift users specify who can access which part of a Swift object. To give a concrete example, consider that a hospital stores its patient records  as  Swift objects. These records should be accessed differently by different personnel. For example, `doctors' can see certain part  while the `billing accountant' can see other part of the record. Our implementation would let the data publisher specify policies expressing who can see which part of the data.

Our prototype implementation is based on JSON formatted documents. We use JSON data because recently JSON has gained immense commercial popularity which is reflected by developments including JSON document database such as MongoDB supported by the OpenStack cloud platform, Twitter's latest API (v 1.1) which supports only JSON data and so on.
