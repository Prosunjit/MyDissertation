\begin{abstract}
	
			There are two major techniques for specifying authorization policies in Attribute Based Access Control (ABAC) models. The more conventional approach is to define policies by using logical formulas involving attribute values. Examples in this category include ABAC$_\alpha$, \hgabac\ and XACML. The alternate technique for expressing policies is by enumeration. Policy Machine (PM) and \twoSortedRBAC{} fall into the later category. In this paper, we present an ABAC model named LaBAC (Label-Based Access Control) which adopts the enumerated style for expressing authorization policies. LaBAC can  be viewed as a particularly simple instance of the Policy Machine. LaBAC uses one user attribute ($\uLabel$) and one object attribute ($\oLabel$). An authorization policy in LaBAC for an action is an enumeration using these two attributes. Thus, LaBAC can be considered as a bare minimum ABAC model.  We show equivalence of LaBAC and \twoSortedRBAC{} with respect to theoretical expressive power. Furthermore, we show how to configure the traditional RBAC (Role-Based Access Control) and LBAC (Lattice-Based Access Control) models in LaBAC to illustrate its expressiveness.
			
			%But  while, PM is more general and thus complex by covering other interesting aspects of access control, LaBAC is more scoped regarding development and progress towards ABAC models.
			
			%In this paper, we conceptualize Attribute Based Access Control (ABAC) with enumerative policies. While doing so, we acknowledge two major approaches for reasoning ABAC policies - flexible-language based policy (e.g. policy in $\abacAlpha$), and fixed-structured enumerative policy (e.g. policy in Policy Machine). Although not officially reported, the NP-completeness property of policy review in flexible-policy models is recognized in academic community which acts as the motivation for designing our enumerative model, LaBAC. Although, policy enumeration of LaBAC is similar to that of Policy Machine, LaBAC is more scoped towards modeling of ABAC, while Policy Machine is more general and thus complex by covering  many other interesting aspects of access control.  In modeling, we visualize LaBAC as a bare minimum ABAC model having only one user attribute ($\uLabel$) and one object attribute ($\oLabel$). We also provide a family of LaBAC models from the basic  to the most advanced one.  We show the expressiveness of our model by configuring different RBAC, LBAC models in the spectrum of LaBAC family. Finally, we discuss merits and demerits of both approaches for modeling policies in the design of an ABAC model.
	
			%In this paper, we conceptualize Attribute Based Access Control (ABAC) assuming finite domains for its components. The consequent of the finiteness of attributes, attribute values and  operators to express condition is that only a finite set of policies can be expressed. Having the same finiteness assumptions, while most other ABAC models use complex policy languages taking advantage of large set of attributes, we take a rather opposite enumerative approach using smallest possible attributes. Our presented Model, LaBAC (Label Based Access Control) uses one user attribute (\uLabel), one object attribute ($\oLabel$) and an enumerative style for representing policies. We show flexibility of the model by configuring LBAC, RBAC and DAC. We later enhance LaBAC using extended policy and accommodating more attribute on both user and object side. We discuss merits and demerits of the two approaches - flexible-language policy vs fixed-enumerative policy. Our initial finding is that policy review in flexible-language model is NP-complete. On the other hand, policy review is polynomial in fixed-enumerative model but it  can result very large even exponential policies.
			%For example, we assume the set of attributes on users and objects, co-domain of each attribute function, the logical operators to form conditional expression, set of actions are all finite. One consequence is that using finite set of elements, our model express


			%In the arena of access control, Attribute Based Access Control (ABAC) is expected to be a super access control model  capable of performing DAC (Discretionary Access Control), MAC (Mandatory Access Control), RBAC (Role Based Access Control) and beyond. In order to meet these expectation, it should also be very flexible, expressive and easily extendable. In this regard, some fundamental work has been done to reveal the power of ABAC models.  Still the state of ABAC is premature and demands for a better understanding from different perspectives.
			
			%In this paper, we present a simple  ABAC model (LaBAC) based on single user attribute (\uLabel) and single object attribute (\oLabel). While one perspective of designing ABAC models (e.g. $\abacAlpha$) is to take advantage of complex policy language to express large number of policies, \textbf{we show that complex language make policy review and policy update NP-complete even for single attribute.} This inspire  us to design fixed-policy model which allows polynomial time  policy review and update. To show flexibility of the model, we show how to configure LBAC, RBAC and DAC* in \labac. We also compare "policyspace"  of different attribute based models. \textbf{ Our finding is that \labac{} can configure any policy expressible with two attributes (one user attribute and one object attribute), whereas  expressiveness of flexible-policy model depends on corresponding policy language.} We also present some possible extensions of \labac including addition of more attributes in the design of \labac{}.  Finally, we develop a proof-of-concept instantiation of \labac{} in the context of OpenStack Swift.
			
			%In this paper, we present a framework for specifying attribute based access control models starting from a very simple model (\labacOneOneOne{}) of one object attribute and one user attribute. We show how to configure MAC, RBAC and DAC* using \labacOneOneOne{}. We progressively generalize \labacOneOneOne{} covering large number of user and object attributes in \labacOneMN{} model. In our framework, we take a fundamentally different approach for specifying attribute based policies. While, all of the existing models take advantages of  complex policy languages to realize expressive power, we maintain a simple and fixed policy language. We show that complex and arbitrary nature of policy language turns policy review functions and policy update into a NP-complete problem. We also specify a framework for understanding wild nature of attributes and we scope our ABAC framework in the specified attribute framework. Finally, as a proof of concept, we have implemented and enforced policies of some of the ABAC models in the context of OpenStack Swift.

			%Attribute Based Access Control (ABAC) models are expected to be flexible and scalable in open distributed environment. The flexibility is achieved by using various attributes to configure fine grained access control policies. This flexibility comes at a cost of potentially large number of policies. Unfortunately, when number of attributes grows, number of possible policies grows exponentially. On the otherhand, individual policy can possibly contain a large number of  attributes, and thus creates a monolithic structure. This situation complicates administration and maintenance of the ABAC policies which we believe hinders its adoption in the industries.
			
			%In this paper, we propose enhancements of ABAC model to keep the flexibility and scalability of ABAC models yet reducing the number of possible policies with reduction in the length of individual policies. In order to reduce policy length, we introduce the concept of \emph{attribute reduction} where  a set of simultaneously used `basic'  attributes are reduced into a much smaller set of `derived' attributes. On the other hand, to reduce number of policies, our model groups individual permission into meaningful permission-set and policies are defined for each permission-set instead of individual permissions.
			
			

\end{abstract} 