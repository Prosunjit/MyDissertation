
\newcommand{\phiu}{\phi_{u}}
\newcommand{\phio}{\phi_{o}}
\newcommand{\phia}{\phi_{a}}
\newcommand{\phip}{\phi_{p}}
\newcommand{\phix}{\phi_{x}}
\newcommand{\phiy}{\phi_{y}}
\newcommand{\userAttrExpr}{UAExpr}
\newcommand{\objectAttrExpr}{OAExpr}
\newcommand{\actionExpr}{AExpr}
\newcommand{\review}{acting\_users}
\newcommand{\simpleReview}{deep\_review}
\newcommand{\isSatisfiable}{is\_satisfiable}
\newcommand{\interpretationY}{I_y}
\newcommand{\interpretationX}{I_x}
\newcommand{\interpretation}{I}
\newcommand{\reductionAlgo}{CircuitSAT2PolicyReview}
\newcommand{\psiu}{\psi_{u}}



\subsection{\sABAC{} as a Conventional ABAC Model}
\sABAC{} model is defined for a finite set of users $U$, objects $O$ and actions $A$. Additionally, it has a set of attribute functions for users and objects. One or more user attributes form UAExpr, one of more object attributes form OAExpr and oen or more actions from AExpr.

	
\newcommand{\policyEval}{\delta}
% Please add the following required packages to your document preamble:
% \usepackage{booktabs}
\begin{table*}
	\centering
	\caption{  \sABAC{} Model} %\vspace*{3pt}
	\label{tab:labac-definition}
	\begin{tabular}{|l|}						
		\hline					
		\multicolumn{1}{|c|}{\underline{\textit{I. \sABAC{} Model } } }\\	
		
	 	- $U, O, A$: set of users, objects and actions \\
	 	- $UA,OA$: set of attribute functions for users and objects respectively.   \\ \hfill	$for_{ua \in UA}, ua: U \to 2^{range(ua)} $ and
	 	$for_{oa \in UA}, oa: O \to 2^{range(oa)} $ \\
	 	- policy: for an action, a policy $\phi_a$ is defined as shown in table \ref{tab:sabac-def}\\
	 	%-  policy: For an action a, a policy \phi_a is defined as shown table ?? \\
	 	 
	 
	 \hline	
	\end{tabular}
	
\end{table*}

%mod

	\newcommand{\UAttrVal}{\text{UAttrVal}}
\newcommand{\OAttrVal}{\text{OAttrVal}}
\newcommand{\Action}{action}
\newcommand{\E}{E}
% Please add the following required packages to your document preamble:
% \usepackage{booktabs}
\begin{table}[]
\centering
\caption{Policy Definition for \sABAC}
\label{tab:sabac-def}
\begin{tabular}{@{}l@{}}
 \hline
	$\phi ::= \phi \land \phi | \ \phi \lor \phi | \ (\phi) | \neg \phi $\\
 
	$\phi ::= \E $ \\
	Terminal Symbols:\\
	$ \E (attr: UA \cup OA, val: range(Attr))$ \\ \hfill  $\to \{true, false\}$ \\
	%$Act(\requestContext, a) \to \{true, false\}$\\
 %\bottomrule
 \\\hline
\end{tabular}
\end{table}

 
	


\subsection{Review Functions in \sABAC{}}
	

\textbf{Interpretation of Attribute Expression}\\
	Informally, an interpretation of an attribute expression  is the set of  $(attr,Value)$ pairs where $attr \in UA \cup OA$ and $Value \subseteq Range(attr)$ for which the attribute expression is evaluated to be true. Formally, we define interpretation as a function $\interpretation$ as follows.

\begin{itemize}
	\item $\interpretation(\phi) \to 2^{(attr: UA \cup OA,  Value: 2^{Range(attr)})}$ and  defined as \\
	$\interpretation(\phi) = \{(attr,Value) | (E(attr,val) \implies true ) \implies (\phi \implies true) \land val \in Value \}$
	
\end{itemize}
	
\noindent \textbf{Example of \sABAC{} Policy and its interpretation} \\
let $ \phi  \equiv \{ E(role, manager) \land \lnot E(role, dir) \land E(type, new) \}$ be a  \sABAC  policy. 	
$I_\phi = \{ (role,  \{ manager\}),$  $(type, \{ new\}), $   $(action, \{ approve\}) \}$ is satisfying interpretation for $\phi $. but $\{ (role,  \{ manager, dir\}) , (type, \{ new\}),$ \\ $(action, \{ approve\}) \}$ is not a satisfying interpretation. \\

\noindent \textbf{Review Function As a Decision Problem}
	
$\simpleReview=\{  \phi$ | exists an assignment of user and object attribute values of policy $\phi$ st. $\phi$ is evaluated true \}  \\
	
%Finally, we define review function $\reviewFunction$ as follows.$R(\phi, input:2^{\interpretation_\phi}) \to 2 ^{\interpretation_\phi}$, defined as  $R(\phi, input) = 2^{\interpretation_\phi} \setminus input$ \\	\\



 
 
% % Please add the following required packages to your document preamble:
 % \usepackage{booktabs}
 \begin{table}
 	\centering
 	\caption{Some review functions for \sABAC{}}
 	\label{tab:review-fun}
 	\begin{tabular}{|l|l|l|}
 		 \hline
 		 \textit{function Name} &  \textit{Input} &  \textit{Output}\\
 		 \hline
 		 \textit{\request} &   \{$\interpretation(\phiu), \interpretation(\phio), \interpretation(\phia)$\}&  $\delta \equiv \{true, false\}$ \\ 		 
 		 
 		 \hline
 		 \textit{UA-query} &   \{$ \interpretation(\phio), \interpretation(\phia), \delta$\}&  $\{\interpretation(\phiu)\}$ \\
		\hline
		\textit{OA-query} &   \{$ \interpretation(\phiu), \interpretation(\phia), \delta$\}&  $\{\interpretation(\phio)\}$ \\
		\hline
		\textit{Act-query} &   \{$ \interpretation(\phiu), \interpretation(\phio), \delta$\}&  $ \{ \interpretation(\phia) \}$ \\
		\hline
		\textit{UO-query} &   \{$ \interpretation(\phia), \delta$\}&  $ \{ \interpretation(\phiu),\interpretation(\phio) \}$ \\
			\hline
		\textit{deep-query} &   \{$\delta$\}&  $ \{ \interpretation(\phiu),\interpretation(\phio),  \interpretation(\phia),  \}$ \\
			\hline
 	\end{tabular}
 \end{table} 
 

  \subsection{Policy review in \sABAC{} is NP-Complete }
	
	we first informally argue that there exists a one-to-one correspondence between a boolean circuit and a policy in \sABAC{} model. Roughly, a boolean circuit is composed of n boolean variables/inputs (each denoted by $x_i$ and having a value from $\{0, 1\}$ )   and one boolean output ($y$) (in general, m outputs. But for our purpose we stick to one output). On the other hand, a \sABAC{} policy is composed of one or more functions of  $E$ which is also evaluated to be either true and false.  While boolean variable uses AND, OR and NOT gates, \sABAC{} policies uses logical $\land, \lor, \lnot$ respectively which corresponds semantically. Analogous to the output of a boolean circuit, evaluation of an \sABAC{} policy is either true or false. Thus we say a boolean circuit and a \sABAC{} policy correspond to each other. 
	
	Satisfiability of a boolean circuit asks for values of each boolean variable, $x_i$ that make the output of the circuit to be one. Similarly,  a review function (for example, in deep_review in Table \ref{table:review-fun}) may ask for possible attribute value assignments that evaluates  a access control request to be true. 
	%Each input ($x_i$) in the boolean circuit, can be thought of a evaluation function ($E$) in the UAExpr and OAExpr. \sABAC 
	
	 \subsubsection{CircuitSAT}
	 

\begin{table}[]
	\centering
	\caption{Boolean Circuit }
	\label{my-label}
	\begin{tabular}{@{}l@{}}
		\toprule
		$\psi ::= \psi \ AND \  \psi | \psi \ OR \  \psi | (\psi) | \ NOT \ \psi $ \\
		 $ \psi ::= x_i$ \\
		Terminal Symbols \\
		$x_i \in \{true, false\} $  \\		
		\bottomrule
	\end{tabular}
\end{table}

\noindent $\CircuitSAT = \{ \psi |$ there exists truth value assignments of each input $x_i$ that satisfy the boolean circuit $\psi$\} \\

	 

	
	 \textbf{NP Completeness of Policy Review} 
\begin{itemize}
	\item $\simpleReview \in NP$
		 given an instance of \simpleReview,   $\phi$  and a certificate $\interpretation$ for $\phi$. We can verify the certificate by evaluating each function $\E \in \phi$ with given attribute and values in $\interpretation$ in linear time. 
		
	\item Algorithm \ref{alg:reduction}  reduces \CircuitSAT{} to \simpleReview

	\item $\psi \in \CircuitSAT \implies \reductionAlgo(\psi) \in \simpleReview$:
		    $\psi  \in \CircuitSAT$ means there is a satisfying assignment of all $x_i \in \psi$. We replace all boolean variable, $x_i$ with evaluation function, $E_i$ resulting $\psiu$. So there must be a satisfying truth assignment of each $E_i$ satisfying $\psiu$. If we use the same (attribute, value) pairs as used in the construction of each $E_i$, we get a satisfying interpretation of attributes, $\interpretationX$ for $\psiu$.
		     %As a result, for $(\psiu \land \E(a_i, val_i), E(a_i, val_i), ( ai, \{val_i\}) )$, there is an interpretation $\interpretationX$ that satisfy $\psiu$. That means, $\reductionAlgo(\psi) \in \simpleReview$. ( because of fact that  $\reductionAlgo(\psi)$ results $(\psiu \land \E(a_i, val_i), E(a_i, val_i), ( ai, \{val_i\}) )$)
	\item $ \phi \in \simpleReview \implies \exists \psi [ \psi \in \CircuitSAT]$:	\\
		   If we replace each $\E_i$ in $\phi$ with $x_i$ resulting $\psi$, $psi$ is still satisfiable because $\phi$ is satisfiable. This implies that $\psi \in \CircuitSAT$. 
		
	\item Algorithm \ref{alg:reduction} runs in polynomial time. 
\end{itemize}
	
	
	\begin{algorithm}
	\caption{ Reduction of Circuit SAT to Policy Review Problem}
	\label{alg:reduction}
	\begin{algorithmic}[1]
		\Procedure{\reductionAlgo ($Circuit \ \psi$)}{}
		\State Replace boolean ops with logical ops.
		\For{each variable  $x$ \Pisymbol{psy}{206} $\psi$ }
		\State replace $x$ with \textbf{\E($a \in UA \cup OA, val \in range(a)$) } 
		\EndFor
		\State let $\psiu$ denote the resulting \userAttrExpr		
		\State return $\psiu$ 	
		\EndProcedure
	\end{algorithmic}
\end{algorithm}

\subsubsection{Policy Review in LaBAC is Polynomial}
% %\newcommand{\E}{E}
% Please add the following required packages to your document preamble:
% \usepackage{booktabs}
\begin{table} 
\centering
\caption{Policy Definition for \elabac{}}
\label{table:sabac-def}
\begin{tabular}{@{}l@{}}
 \toprule	
	$\phi ::= \phi \land \phi  $\\
	$\phi ::= \E $ \\	 
	Terminal Symbols\\
	$ \E (label: UA \cup OA, val: range(label) \cup NUL \cup NOL)  \to$ \\ \hfill $\{true, false\}$ \\
 \bottomrule
\end{tabular}
\end{table}