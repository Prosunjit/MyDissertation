%\section{\hlabac{} in Policy Machine}
\label{sec:pm}


In this section, we show how \eapABAC{} can be presented as a simple instance of Policy Machine (PM) \cite{policy-machine}. In order to do so, we first define Policy Machine Mini ($\pmMini{}$) - a step down version of PM sufficient enough for our purpose. We then configure \hlabac{} in $\pmMini{}$.

\subsubsection{Policy Machine$_{mini}$ ($\pmMini{}$)}

$\pmMini{}$ is a sufficiently reduced version of Policy Machine (PM).  For example, while PM uses four basic relations namely Assignment, Association, Prohibition and Obligation, $\pmMini$ includes only the first two of these. Similarly, PM manages both resource operations and administrative actions but $\pmMini$ is limited to managing operation on resources only.  Additionally, \textit{Policy Class}, an important concept in PM for combining multiple policies, is not considered in $\pmMini$. 




Definition of $\pmMini{}$ is shown in Table \ref{tab:poilcy-machine-mini}. In $\pmMini{}$ users, objects, operations and processes are denoted by set $U, O, OP$ and $P$ respectively. $UA$ and $OA$ represent the finite sets of user attributes and object attributes. The definition of attributes in $\pmMini$ is different than the definition of attributes in most other models. While typically attributes are used as (attribute, value) pairs, $\pmMini$ uses attributes  as containers for users, objects and other attributes (constraints apply). For example, a user can be assigned to a user attribute $ua_i$ which can further be assigned to another user attribute $ua_j$. Same type of assignment applies for object and object attributes. User (or user attribute) to user-attribute  assignments and object (or object attribute) to object-attribute assignments are captured by the \textit{\assignment}{} relation which must be acyclic and irreflexive. On the other hand, the \textit{\association}{} relation is like a grant relation. The meaning of $(ua,\{a\},oa) \in \textit{\association}$ is that users contained in $ua$ can perform operation $a$ on objects contained in $oa$. Containment of users and objects can be transitive which is specified by the $\assignmentPlus{}$ relation. The decision function  $\decisionFunction(p,op,o)$ allows a process, $p$ (running on behalf of a user, $u$) to perform an operation, $op$ on an object, $o$ if there exists an entry, $(ua,\{op\},oa)$ in \textit{\association}{} relation  where $ua$ transitively contains $u$ and $oa$ transitively contains $o$.

\newcommand{\processUser}{process\_user}


% Please add the following required packages to your document preamble:
% \usepackage{booktabs}
\begin{table}
	\centering
	\caption{ $\pmMini$ definition} %\vspace*{3pt}
	\label{tab:poilcy-machine-mini}
	\begin{tabular}{|l|}						
		\hline					
		\multicolumn{1}{|c|}{\underline{\textit{I. Basic sets and relations }}}\\				 
		 - $U, O, OP$ and $P$ (set of users, objects, operations \\ \hfill  and processes resp.) \\ 
		 - $UA, OA$ (set of user and object attributes) \\  
		 - $AR$ (set of access rights).   In $\pmMini$, $AR=OP$ \\
		 - $\processUser: P \to U$\\	 
		
		\\ \multicolumn{1}{|c|}{\underline{\textit{II. Assignment and association relations}}} \\
			- $\assignment \subseteq (U \times UA) \cup (UA \times UA) \cup (O \times OA)$ \\ \hfill $\cup (OA \times OA),$  an irreflexive, acyclic relation \\
	 
	
		- $\association \subseteq UA \times 2^{AR} \times OA$ \\
	 
		 \\ \multicolumn{1}{|c|}{\underline{\textit{III. Derived relations}}} \\
	 	
		 - $\assignmentPlus$, transitive closure  of $\assignment$   \\		 
	 
	 	
	 
	 	\\ \multicolumn{1}{|c|}{\underline{\textit{IV. Decision function}}} \\
	 	% - $\decisionFunction(p,a,o) = \exists(ua, ars, oa)$  $\in \associationPolicy$   \\ \hfill $[ (u,ua) \in \assignmentUAUA \land (o,oa) \in \assignmentOAOA \land a \in ars]$ 
	 	 
	 	- $\decisionFunction(p,op,o) $ =\\ \hfill $\exists oa \in OA, \exists ua \in UA, \exists u \in U $ \\ \hfill $[(ua,\{op\}, oa) \in \association \land$ \\ \hfill $ (u,ua) \in \assignmentPlus\land (o,oa) \in \assignmentPlus \land $  \\ \hfill$\processUser(p) = u]$ 	
		 	
 		\\ \hline	
	\end{tabular}	

	
\end{table}


A process in $\pmMini{}$ simply inherits all attributes of the creating user. Thus $\pmMini{}$ lacks the ability to model sessions, since there is no user control over a process's attributes.  Note that PM achieves this effect through obligation and prohibition relations \cite{INCITS526}. A complete and detailed model of Policy Machine can be found here \cite{INCITS526,policy-machine}.

  % In modeling sessions,  $\pmMini$ lacks the ability to model sessions like they are modeled (Section \ref{sec:session-management}) in LaBAC. Note that PM achieves these through Obligation and Prohibition relations \cite{INCITS526}. For curious readers, a complete and detailed model of Policy Machine can be found here \cite{INCITS526,policy-machine}.

 \subsubsection{Configuring \hlabac{} in $\pmMini{}$}
 
 \begin{table}
	\centering
	\caption{ \hlabac{} in $\pmMini{}$ } %\vspace*{3pt}
	\label{tab:labac-in-policy-machine}
	\begin{tabular}{|l|}						
		\hline					
			\multicolumn{1}{|c|}{\underline{\textit{I. \hlabac{} components}}} \\ \\
			-  $U_L, O_L,  A, S$ (set of users, objects, actions  and sessions  resp.) \\ 
			- $UL, OL, ULH,  OLH$ (uLabel values, oLabel values,  uLabel and \\ \hfill oLabel value hierarchy  resp.) \\		  
			-  $\uLabel: U \to 2^{UL}$, $\oLabel: O \to 2^{OL}$ \\
			-  $\Policy_{a}$, authorization policy for action $a \in A$\\
			- $\creator: S \to U$\\
		   
		\\ \multicolumn{1}{|c|}{\underline{\textit{II. Construction}}}\\	\\
		- $ U=U_L, O=O_L, OP=A, P = S $\\
		- $\processUser(s) = \creator(s)$, for $s \in S$\\
		- $UA = UL, OA=OL$ \\		
		- $\assignment$ = $\{(u,ul) | ul \in \uLabel(u) \} \cup$   
										$\{(ul_i, ul_j) | ul_i \udominate ul_j \} \cup$ \\ \hfil
										$\{(o,ol) | ol \in \oLabel(o) \}\cup $  
										$\{(ol_i, ol_j) | ol_j \udominate ol_i \}$ \\
		- $\association= \{ (ul, a, ol) | \exists (ul,ol) \in \Policy_a$  $ \land \Policy_a \in \Policy \}$ 

		\\ \hline	
	\end{tabular}	

	
\end{table}

 
As $\pmMini{}$ lacks the ability to manage sessions, here we present a mapping from $\pmMini{}$ to \hlabac{} without session management. In the mapping, users, objects, actions and sessions in \eapABAC{} are directly mapped to users, objects, operations and processes in $\pmMini{}$.  User-label values and object-label values in \eapABAC{} correspond to $UA$ and $OA$ respectively. Additionally,  user to user-label value assignments, object to object-label value assignments,  $ULH$ and $OLH$ in \hlabac{} are mapped to the \assignment{} relation. Finally, each tuple in each policy in \eapABAC{} is contained in the \association{} relation. A mapping from  $\pmMini{}$ to \hlabac{} is given in Table \ref{tab:labac-in-policy-machine}. 
