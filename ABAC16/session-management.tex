


%\subsection{Functional Specification}
\label{sec:session-management}
\begin{table*}
\centering
\caption{User-level session functions in \clabac{}}
\label{tab:session-management}
\begin{tabular}{|l|l|l|}
	\hline
\textbf{Fuction}                                                              & \textbf{Condition} & \textbf{Updates} \\ \hline

%\begin{tabular}[c]{@{}l@{}}\createSession\\ (u:U, s:S, values)\end{tabular} & \begin{tabular}[c]{@{}l@{}} $u \in U \land s \not \in S \land values \subseteq \uLabel(u) \setminus \{\createReq, \removeReq\}\  $ \\ $\createReq \in \uLabel(u) \land $ \\ $\exists\Policy_{\createSession} \equiv \{(\createReq, \sessionOL) \}\in \Policy$ \\ \end{tabular} &  \begin{tabular}[c]{@{}l@{}}$S' = S \cup \{s\}$ \\ $\creator(s) = u \land \sessionLabels(s)=value $ \end{tabular} \\\hline
%\multicolumn{3}{c}{\textbf{\textit{User level functions}}}\\ \hline

\begin{tabular}[c]{@{}l@{}}$\createSession$\\ $(u:U, s:S, values:2^{UL}$)\end{tabular} & \begin{tabular}[c]{@{}l@{}} $u \in U \land s \not \in S \land values \subseteq \uLabel(u)$  \end{tabular} &  \begin{tabular}[c]{@{}l@{}}$S' = S \cup \{s\}$,  $\creator(s) = u ,$ \\ $ \sessionLabels(s)=value $ \end{tabular} \\\hline


\begin{tabular}[c]{@{}l@{}}$\deleteSession$\\ $(u:U, s:S)$\end{tabular}   & \begin{tabular}[c]{@{}l@{}} $u \in U \land s  \in S \land creator(s)=u  $ \end{tabular}                   & \begin{tabular}[c]{@{}l@{}}$S' = S \setminus \{s\}$ \end{tabular}                             \\\hline


\begin{tabular}[c]{@{}l@{}}$\assignValues$\\ $(u:U,s:S, values:2^{UL}$)\end{tabular} &  \begin{tabular}[c]{@{}l@{}} $u \in U \land s  \in S \land creator(s)=u \land  values \subseteq \uLabel(u)$ \end{tabular} &   \begin{tabular}[c]{@{}l@{}} $\sessionLabels(s)= \sessionLabels(s) \cup values $ \end{tabular}                \\ \hline


\begin{tabular}[c]{@{}l@{}}$\removeValues$\\ $(u:U,s:S, values:2^{UL}$)\end{tabular} &  \begin{tabular}[c]{@{}l@{}} $u \in U \land s  \in S \land creator(s)=u \land   values \subseteq \uLabel(u)$ \end{tabular} &   \begin{tabular}[c]{@{}l@{}} $\sessionLabels(s)= \sessionLabels(s) \setminus values $ \end{tabular}                \\ \hline

%\begin{tabular}[c]{@{}l@{}}$\createObject$\\ $(s:S, o:O, values:2^{OL}$)\end{tabular} &  \begin{tabular}[c]{@{}l@{}} $s \in S \land o \not \in O \land $ \\$ f_{\createObject}(s,o, values) $ \end{tabular} &   \begin{tabular}[c]{@{}l@{}} $O' = O \cup \{o\}, \oLabel(o) = values$ \end{tabular}                \\ \hline

%\begin{tabular}[c]{@{}l@{}}$\assignLabels$\\ $(s:S, o:O, values:2^{OL}$)\end{tabular} &  \begin{tabular}[c]{@{}l@{}} $s \in S \land o  \in O \land $ \\$ f_{\assignLabels}(s,o, values) $ \end{tabular} &   \begin{tabular}[c]{@{}l@{}} $ \oLabel(o) = \oLabel(o) \cup values  $ \end{tabular}                \\ \hline

%\begin{tabular}[c]{@{}l@{}}$\removeLabels$\\ $(s:S, o:O, values:2^{OL}$)\end{tabular} &  \begin{tabular}[c]{@{}l@{}} $s \in S \land o  \in O \land $ \\$ f_{\removeLabels}(s,o, values) $ \end{tabular} &   \begin{tabular}[c]{@{}l@{}} $ \oLabel(o) = \oLabel(o) \setminus values  $ \end{tabular}                \\ \hline


%\begin{tabular}[c]{@{}l@{}}\removeValues\\ (u:U,s:S, values:2)\end{tabular} &  \begin{tabular}[c]{@{}l@{}} $u \in U \land s  \in S \land $  $\creator(s) = u \land $ $ \removeReq \in \uLabel(u) $ \\ \end{tabular} &   \begin{tabular}[c]{@{}l@{}} $\sessionLabels(s)= \sessionLabels(s) \setminus values $ \end{tabular}                \\ \hline
                 
\end{tabular}
\end{table*}



\eapABAC{} allows users to create or destroy sessions, and assign/remove values from an existing session. Table \ref{tab:session-management} presents user-level \textit{\sessionLabels{}} functions for managing sessions in \clabac{}. Each function is presented with formal parameters (given in the first column), necessary preconditions (in the second column) and resulting updates (in the third column).  The function $\createSession()$ creates a new session with given values, $\deleteSession()$ deletes an existing session, $\assignValues()$ assigns values in an existing session, and $\removeValues()$ removes values from an existing session. 

%First column in the table shows function names along with formal parameters, second column defines precondition which must be satisfied for the function to be executed. The third column describes updates in the LaBAC sets and relations once corresponding function is executed.

In \hlabac{}, we modify condition of the session functions from Table \ref{tab:session-management} to accommodate  that in a session created by a user, he can choose from the values he is assigned to or junior values. The modified conditions are given in Table \ref{tab:session-in-hlabac}. We specify an additional condition with each session function  in \consLabac{} and \labacOneOneOne{}.  For example, with $\createSession()$,  we specify a boolean function $f_{\createSession}()$ as additional precondition which must also be true. The definition of these boolean functions are  open-ended to be able to configure any session constraints. The difference between session functions in \consLabac{} and \labacOneOneOne{} is that the former does not consider hierarchy on user-label values whereas the later does. Table \ref{tab:session-in-consLabac} and \ref{tab:session-in-labacOne} show session functions in \consLabac{} and \labacOneOneOne{} respectively. Table \ref{tab:example-f-create-session} presents some examples of constraints specified with $f_{\createSession}()$ function.  \textit{Example 1} uses an enumerated policy, $\Policy_{\createSession}$. It specifies that in order to create a session and assign values to the session, a user must be assigned to value $\createReq$. \textit{Example 2} enforces the constraint that no more than one conflicting $\uLabel$ values can be activated in a session. \textit{Example 3} imposes that a user cannot have more than some bounded number of sessions.

Note that creation and deletion of objects, updating object-label values by sessions are outside the scope of \eapABAC{} operational models presented here. One reason behind is that, \eapABAC{} only focuses on attributes. It can be extended to include object creation and modification along the line of $ABAC_{\alpha}$ \cite{abacAlpha}. See Table \ref{tab:lbac-in-labac} for example.

\begin{table} 
\centering
 \captionsetup{justification=centering}
\caption{Session functions in \hlabac{} \newline (condition of session functions modified from Table \ref{tab:session-management} ) }
\label{tab:session-in-hlabac}
\begin{tabular}{|l|l|} \hline
\textbf{Function} & \textbf{Modified condition} \\ \hline
  $\createSession{}$      & \begin{tabular}[c]{@{}l@{}}  $u \in U \land s \not \in S \land values \subseteq$ \\ \hfill$  \{ ul' | \exists ul \udominate ul' [ ul \in \uLabel(u)] \}$ \end{tabular}             \\ \hline
	$\deleteSession{} $        & $u \in U \land s  \in S \land creator(s)=u  $                     \\ \hline
     $\assignValues{}$    &     \begin{tabular}[c]{@{}l@{}}  $u \in U \land s  \in S \land creator(s)=u \land values $ \\ \hfill$  \subseteq \{ ul' | \exists ul \udominate ul' [ ul \in \uLabel(u)] \}$ \end{tabular}                \\ \hline
 $\removeValues{}$    &     \begin{tabular}[c]{@{}l@{}}  $u \in U \land s  \in S \land creator(s)=u  \land values $ \\ \hfill$  \subseteq \{ ul' | \exists ul \in  \uLabel(u) \land ul \udominate ul' \}$ \end{tabular}                 \\ \hline
\end{tabular}
\end{table}



\begin{table}[]
\centering
 \captionsetup{justification=centering}
\caption{Session functions in  \consLabac{} (condition added with session functions from Table \ref{tab:session-management})}
\label{tab:session-in-consLabac}
\begin{tabular}{|l|l|} \hline
\textbf{Session function} & \textbf{Additional condition} \\ \hline
   $\createSession{}$      & $\land f_{\createSession}(u,s,values)$               \\ \hline
	$\deleteSession{} $        & $\land f_{\deleteSession}(u,s)$                     \\ \hline
     $\assignValues{}$    &      $\land f_{\assignValues}(u,s,values)$                \\ \hline
 $\removeValues{}$    &      $\land f_{\removeValues}(u,s,values)$                \\ \hline
\end{tabular}
\end{table}

\begin{table}[]
\centering
 \captionsetup{justification=centering}
\caption{Session functions in  \labacOneOneOne{} (condition added with session functions from Table \ref{tab:session-in-hlabac})}
\label{tab:session-in-labacOne}
\begin{tabular}{|l|l|} \hline
\textbf{Session function} & \textbf{Additional condition} \\ \hline
   $\createSession{}$      & $\land f_{\createSession}(u,s,values)$               \\ \hline
	$\deleteSession{} $        & $\land f_{\deleteSession}(u,s)$                     \\ \hline
     $\assignValues{}$    &      $\land f_{\assignValues}(u,s,values)$                \\ \hline
 $\removeValues{}$    &      $\land f_{\removeValues}(u,s,values)$                \\ \hline
\end{tabular}
\end{table}



\begin{table}
	\centering
 \caption{Examples of $f_{\createSession}(u, s, values)$}
 \label{tab:example-f-create-session}
	\begin{tabular}{|l|}
		\hline	                                                                                           	
		%\multicolumn{1}{|c|}{\underline{\textit{Examples in \consLabac{}/\labacOneOneOne{}:}}}\\                	
		\multicolumn{1}{|l|}{{\textit{Example 1. using LaBAC policy:}}}\\
		
		$\exists \createReq \in \uLabel(u) \land$ \\$ \exists \Policy_{\createSession} \equiv \{(\createReq, \sessionOL) \}\in \Policy$ \\
		
		\\ \multicolumn{1}{|l|}{{\textit{Example 2. using \labacOneOneOne{} session constraint CSL:}}}\\
		 $|values \cap OneElement(CSL)| <= 1$ \\
		 
		 \\\multicolumn{1}{|l|}{{\textit{Example 3. using cardinality  constraint on sessions:}}}\\
		 $|\{s | \creator(s)=u\}| <= 10$ \\
		 \hline
		\end{tabular}  

\end{table}

